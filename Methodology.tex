\chapter{Methodology}

Analyzing the source code of the program will give a good view of what is inside the program, and potentially find out where it is suitable for GPU parallel computation so that it can be implemented in the future. To investigate spots in RegCM that are likely to be benifitted from GPU parallelism, profiling the software to locate what part of it takes the most computation time is the initial and crucial step. Next, data dependencies needs to be detected in order avoid paralleling those section and hurts the performance of the program, or potentially break it. Finally, CPU thread-level is used to run RegCM as a proof of concept to see if there are any improvement in performance. While the initial goal of the internship is to have a demo of RegCM running parallel on GPU, but because of the time constrain, the best can be done was to evaluate the possibility of GPU parallelization. \\

\section{RegCM Installation}

As stated in the previous chapter, the main purpose of the intership is to investigate possible utilization of GPU parallelism within the Regional Climate Model. In order to achieve this goal, the source code of RegCm is required for analyzing the software and modify it for the demand of the intership. Furthermore, because the cluster used in REMOSAT lab is usually busy with running simulations, it would be best if RegCM can be installed in the ICTLab for easy accessibility. \\
~\\
Thankfully, RegCM is a open-source software, a term stands for softwares that are publicly accessible and anyone can share and/or revamp it. It is publicly available at: \url{http://gforge.ictp.it/gf/project/regcm/} . In this internship, version 4.3.5.6 is used because it is one of the most stable version and it is being ran by the REMOSAT laboratory. \\
~\\
The source code contains 3 main parts, 2 of those are for programs that pre-process the input data set and post-process the output of the simulation. The last part, and also the most important part, is the source code for RegCM, containing the main program of RegCM with its support modules. The simulation software is written in Fortran, a programming language that convinient for numeric and scientific computing. \\
~\\
Before the compilation of RegCM's source code, its prerequisite software need to be installed. Those are Intel Fortran compiler, Hierarchical Data Format 5 (HDF5) and Network Common Data Form(NetCDF)-C 4 and NetCDF-Fortran 4. Moreover, for NetCDF, both C and Fortran version, to function properly, HDF5 is needed. In addition, supporting libraries, zlib and szip, are also mandatory. Finally, the Operating System used in this project is Debian, a Linux-distribution. \\
~\\
To start off RegCM installation process, Intel Fortran compiler needs to be installed first of all. This compiler is developed by Intel and is available in the Intel Parallel Studio XE development tools, designed specifically for High Performance Computing. This package is a commercial software, but Intel also offers free licenses for students and researchers. Simply follow the installation guide, enter a given product key, choose where to install and wait for the installer to complete. \\
~\\
After the installer is done, to complete the installation progress, enter the directory \\ \verb|intel_dir/bin/| and run 3 scripts, iccvars.sh, ifortvars.sh and compilervars.sh with parameter being the current system architecture, \verb|ia32| for 32-bit systems and \verb|intel64| for 64-bit systems. For example: 
\begin{center}
\begin{BVerbatim}
source compilervars.sh intel64
\end{BVerbatim}
\end{center}
Doing this will export the location contains the compiler executable files to the \verb|PATH| variable, thus enable the user to run these compilers without having to type the path of the compiler everytime. Lastly, export these variables: \\
\begin{center}
\begin{BVerbatim}
export CC=icc
export FC=ifort
\end{BVerbatim}
\end{center}
so that the default compiler for the system will those of Intel's. \\
~\\
Next, zlib and szip should be compiled and built, respectively. Both of these libraries are required for installing HDF5 and NetCDF later on. They are available at \url{https://zlib.net/} and \url{http://www.compressconsult.com/szip/}. In order to install these libraries, simply run the following commands
\begin{center}
\begin{BVerbatim}
./configure --prefix=/directory/to/install
make
make install
\end{BVerbatim}
\end{center}
each time for each library. The \verb|--prefix=/directory/to/install| is for installing libraries in a different folder other than the default \verb|/usr/local/| folder. \\
~\\
HDF5's source code can be obtained at \url{https://support.hdfgroup.org/HDF5/release/obtain5.html}. The version was used in the ICTLab's cluster was HDF5-1.8.16. After the source code was extracted from the archive, the installation progress takes 3 commands, just like with zlib or szip:
\begin{center}
\begin{BVerbatim}
./configure --prefix=/directory/to/install --with-zlib=/zlib-dir
	--with-szip=/installed/szip/dir --enable-parallel
make
make install
\end{BVerbatim}
\end{center}
Nonetheless, it is crucial to state in the configure command where were szip and zlib installed, if they were not installed in the default folder. The \verb|--enable-parallel| paramemter will tell the configuration process to look for MPI-IO and MPI support files, which was installed previously along with Intel Parallel Studio. \\
~\\
To install the last prerequisite, and undoubtedly the most important one, NetCDF, there are a couple of things to note. First, there are two versions of NetCDF, one using C interface and the other uses Fortran, and both are required. Second, unlike HDF5's install process where the user can state where the libraries by configuration's parameter, NetCDF requires the user to specify libraries's path to \verb|CPPFLAGS|, \verb|LDFLAGS| and \verb|LD_LIBRARY_PATH|. Source code for NetCDF is available at \url{https://www.unidata.ucar.edu/downloads/netcdf/index.jsp}. Again, the version of NetCDF-C and NetCDF-Fortran was used in this project are those that used by REMOSAT, NetCDF-C 4.4.0 and NetCDF-Fortran 4.4.3. NetCDF is installed in the following order: NetCDF-C, then NetCDF-Fortran. \\
~\\
To compile and build NetCDF-C, the initial step, as stated previously, libraries's path needs to be added to environement variables
\begin{center}
\begin{BVerbatim}
export CXX=icpc
export CPPFLAGS="-I/zlib-dir/inlcude -I/hdf-dir/include"
export LDFLAGS="-L/zlib-dir/lib -L/hdf-dir/lib"
export LD_LIBRARY_PATH=/zlib-dir/lib:/hdf-dir/lib:$LD_LIBRARY_PATH
\end{BVerbatim}
\end{center}
After setting environment variables is complete, proceed to configure and install NetCDF-C
\begin{center}
\begin{BVerbatim}
./configure --prefix=/directory/to/install
make
make install
\end{BVerbatim}
\end{center}
With NetCDF-Fortran, the environment variables needs to be updated with NetCDF-C's path
\begin{center}
\begin{BVerbatim}
export CPPFLAGS="-I/zlib-dir/inlcude -I/hdf-dir/include 
	-I/netcdf-c-dir/include"
export LDFLAGS="-L/zlib-dir/lib -L/hdf-dir/lib 
	-L/netcdf-c-dir/lib -lnetcdf"
export LD_LIBRARY_PATH=/netcdf-c-dir/lib:$LD_LIBRARY_PATH
\end{BVerbatim}
\end{center}
Following that, start the install process, which is simillar to NetCDF-C's, and once finished, all needed compoments are installed and RegCM are ready to be configured and built. \\
~\\
In order to install RegCM, these compiler variables have to be exported:
\begin{center}
\begin{BVerbatim}
export I_MPI_CC=icc
export I_MPI_CXX=icpc
export I_MPI_FC=ifort
export I_MPI_F77=ifort
export I_MPI_F90=ifort
\end{BVerbatim}
\end{center}
Previously mentioned, RegCM has full support for Message Passing Interface, or MPI, so MPI compilers are necessary. Because the standard compiler using to build RegCM are those of Intel's, the MPI compiler are required to match those as well, or the installation process won't be successful. The installation procedure follows the above normally, though no prefix for installation folder is needed, since all the compiled binary files of RegCM are stored within the newly created \verb|bin| folder inside the source code folder. \\
~\\
The final step of the RegCM Installation Process is to create a working environment for RegCM. This starts with making a new folder to hold the binary files, input and output of simulations. Next, 2 more folders need to be inside the newly created folder, one to contain the pre-processed data for the input of RegCM, the other is the output directory of the simulation. Lastly, a symlink is formed to link the current folder to the bin folder of RegCM. For example:
\begin{center}
\begin{BVerbatim}
ln -sf /RegCM-source-code-folder/bin
\end{BVerbatim}
\end{center}
This link will allow the current folder to contain a reference to the \verb|bin| folder that was created earlier during the installation process. \\
~\\
During initial installation of the software in the internship, some difficulties was met. There was a lack of in-depth information on how to install the software and its prerequisites. The User Manual provided by the software's author did not specify what compiler should be utilized, what version of the dependencies is needed. At first, the programs was installed with standard Linux C and Fortran compilers but many errors occurred along the way. Through many trial-and-errors and with the help of the REMOSAT laboratory, RegCM was finally installed into the cluster system of USTH ICTLab, though it was very time-consuming and took the most ammount of time in the internship. Nonetheless, it was worth putting effort into to ensure a proper working environment. \\

\section{Profiling RegCM}
The first step, and arguably the most crucial step of all, is to profile the Regional Climate Model, to see which part of the program takes the most out of execution, and from there on, as a starting line, dig in further into the program, repeat the process to find spots that are suitable for implementing parallelization. Characteristic that should be on the look out are for-loops with high number of iterations and simple mathematic computation. \\
~\\
To profile the simulation, a subroutine is written to record the current in millisecond and put it at just before the start and right after the end of a designated section of code that need to be profiled to note down the start time and finish time of that section. The time duration it takes to execute that section is the subtraction of the two above values. \\
~\\
Fortran provides a subroutine to retrieve information on computer's date and time in the form of an array, containing data on current year, month, date, time zone, hour, minute, second and millisecond. The subroutine is written as follow:
\begin{center}
\begin{BVerbatim}
subroutine time_in_ms(rTime)
    integer(ik8) :: values(8)
    real(rk16) :: rTime
    call date_and_time(values=values)
    !Hours to minutes  
    rTime = ( values(5) ) * 60
    !Minutes to Second
    rTime = ( rTime + values(6) ) *60
    !Second to ms 
    rTime = ( rTime + values(7) ) *1e3
    !adding ms 
    rTime = rTime + values(8)
 end subroutine time_in_ms
\end{BVerbatim}
\end{center}
In order to get the current time in millisecond, a coversion from other time units to millisecond is needed. Under the premise that no subroutine in RegCM should take more than 24 hours for one run, there are only 3 time units that require to be converted, hour, minute, and second and add all of them up to get the final result.  \\
~\\
In the source code, the file named \verb|mod_regcm_interface.F90| contains the main program of RegCM, initializing the simulation, calling other subroutines from other modules for calculations. These computation of RegCM simulation can be seen in \verb|RegCM_run| subroutine, thus a good place to start. In this subroutine, there is a while-loop, running steps of simulation's duration, containing calls for computing model's tendencies, structure's manipulation, solar declination angle, new boundary conditions reading and filling boundary with input values and tendencies. Measuring the execution time of these subroutines that was called by the main program will give a general look at what takes the most time. The table below shows how the profiling turned out for different simulation's duration: \\
\begin{table}[H]
\centering
\resizebox{\textwidth}{!}{%
\begin{tabular}{@{}lrrrrrrrr@{}}
\toprule
\multirow{2}{*}{} & \multicolumn{2}{c}{2 months}         & \multicolumn{2}{c}{4 months}         & \multicolumn{2}{c}{6 months}         & \multicolumn{2}{c}{8 months}         \\ \cmidrule(l){2-9} 
                  & Total Exe. Time(ms) & Percentage(\%) & Total Exe. Time(ms) & Percentage(\%) & Total Exe. Time(ms) & Percentage(\%) & Total Exe. Time(ms) & Percentage(\%) \\ \cmidrule(r){1-1}
Tendencies        & 1723176             & 94.38          & 3557137             & 94.39          & 5379873             & 94.39          & 7222206             & 94.38          \\
Split Mode        & 92382               & 5.06           & 188920              & 5.01           & 286476              & 5.03           & 384908              & 5.03           \\
Solar             & 1                   & 5.48E-05       & 5                   & 1.33E-04       & 6                   & 1.05E-04       & 8                   & 1.05E-04       \\
New Boundary      & 523                 & 0.03           & 1159                & 0.03           & 1718                & 0.03           & 2352                & 0.03           \\
Fill Boundary     & 1751                & 0.10           & 16021               & 0.43           & 23906               & 0.42           & 32484               & 0.42           \\
Misc.             & 8041                & 0.44           & 5167                & 0.14           & 7832                & 0.14           & 10397               & 0.14           \\
RegCM             & \multicolumn{2}{r}{1825874}          & \multicolumn{2}{r}{3768409}          & \multicolumn{2}{r}{5699811}          & \multicolumn{2}{r}{7652355}          \\ \bottomrule
\end{tabular}%
}
\caption{Result of profiling different subroutines in different simulation duration}
\end{table} 
~\\
It can be seen easily that calculating tendencies for the model takes a tremendous amount of time, occupied aroung 90\% of the whole simulation time. This is a clear sign that the subroutine responsibles for these calculations should be investigated further. \\
~\\
Inside \verb|tend| subroutine of \verb|mod_tendency.F90| file, which handle the model's tendencies calculation, there are 70 main for-loops, all of which are nested with other different for-loops. These main for-loops will be the focus of profiling further \verb|tend| subroutine, as they fit the above wanted characteristics. Again, the profiling process is done much like the same with the first one. \\
~\\
However, things are a little differnet in this case because of 2 key reasons. First, the \verb|tend| subroutine is looped thousand of time, based on the model's simulation time, thus generates lots of data when logging to see the execution time of the loops. Second, there are some loops in this subroutine that are not executed in the process, one may execute in an iteration of the program while not being touched in other iterations. Because of this, it is hard to keep track of all these for-loops, in which knowing when they will occur in the massive amounts simulation's iterations. In order to overcome this problem, 10 profiling results belong to 10 iterations of RegCM will be choosen randomly, and the average execution time of each for-loops inside \verb|tend| will be calculated based on the chosen 10 to see which take the most time. Below are the result: \\
\begin{table}[H]
\centering
%\resizebox{\textwidth}{!}{%
\begin{tabular}{@{}rr@{}}
\toprule
Loop No. & Average Execution Time(ms) \\ \midrule
48       & 2.7                        \\
60       & 1.4                        \\
52       & 1.3                        \\
19       & 0.7                        \\
26       & 0.7                        \\ \bottomrule
\end{tabular}%
%}
\caption{Top 5 most time consuming for-loops in tend subroutine on average}
\end{table}
~\\
The result shows interesting information. While these 5 for-loops are those that take the most time to execute, their average time is very small, capping at only 2.7 ms. Add in the fact that tendencies has the most performance time, these numbers mean that these loops, while big in number of iterations, take very little time to complete. \\
~\\
These 5 for-loops that was shown in the result will be choosen to investigate possible data dependencies, then to try parallelize them to see if there are any improvement in performance. However, one thing to note is that since the number of iterations is enormous, 10 results from randomly choosen iteration might be not enough to clearly asset this problem. Further and deeper inspection of profiling is needed in the future. \\

\section{Investigate possible Data Dependencies}

Data Dependencies, or Data Hazards means that an instruction requires data that is the result of a previous instructrion. Detecting Data Dependencies is a crucial task in implementing parallelization, since data dependencies can harm the parallel computing's performance. The outcome if Data Dependencies is not handled properly is that the parallel program will perform worse since there are parts of the program that has to wait for data to be produced by previous section, thus making parallelization useless while processor still has to schedule the work. \\
~\\
After examined the source code of the 5 fore-mentioned for-loops, it appeared to the naked eye that none of them has any data dependencies. All the loops seem to not using any input that relies on the result produced in previous iterations. However, to be fully confident that there exists no data hazards, inspecting the source code is insufficient, especially with code from other authors since the one doing code's analysis often is foreign to the source code and cannot be sure of the intention of the author. Investigation of this subject will be clearer once these for-loops are parallelize using OpenMP, which will be cover in the next section. This action will allow the observation of each loop's behavior to draw a better conclusion. \\

\section{Open Multi-Processing - OpenMP}

Open Multi-Processing, or OpenMP for short, is an API(Application Programming Interface) for C/C++ and Fortran, contains compiler directives, libraries routine and environment variables that enable high-level parallelism~\cite{ompfaq}. The selling point of OpenMP is that while most of the time, without it, programmers struggle to distribute workload in processors, and to solve the problem of non-standardization of processors's clock speed, the API simplify these programming processes, giving specific instructions automatically to the processors, thus reduced the work need to be done when parallelizing a program. Because of its ease of parallel implementation, existing serialize code can be parallelize with little effort to test parallel compatability of sections of code~\cite{omp_intro}. \\
~\\
%add OMP directive, and use compiler flag and link library
To use OpenMP Parallelization API, simply add the OpenMP directives to sections that need to be parallelize. In this project, the 2 directives that were used are: 
\begin{center}
\begin{BVerbatim}
!$OMP PARALLEL DO
!$OMP END PARALLEL DO
\end{BVerbatim}
\end{center}
These 2 directives were put, respectively, at the beginning and the end of each for-loop, telling the compiler that these sections are to be ran in parallel. Lastly, to compile code that use OpenMP, add the compiler flag and link the library of OpenMP when compiling the code. Depends on the compiler, the flag can varies. Here, the OpenMP flag for Intel compiler is \verb|-fopenmp|. Linking OpenMP library are more uniform by adding \verb|-liopm5| before compilation. \\
~\\
%talk about those loops that failed
It is interesting to note that adding OpenMP directives to loop number 19 and number 26 in \verb|tend| subroutine, RegCM will crash and generate fatal error. It is suspected that these 2 loops has data dependencies of their own that causes the crash to happen. For further testing of potential parallelization of RegCM, the 2 mentioned loops were removed from the scope of the test to ensure RegCM functions normally. 