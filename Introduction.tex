\chapter{Introduction}

\section{Context and Motivation}

Climate models are representation of factors that have impacts on climate, described by mathematical equations based on the law of physics. Main factors that can affect the climate are the atmosphere, oceans, living creatures, land exterior, ice, and energy produced by the Sun. Together with other sophisticated elements such as cloud, rainfall, water evaporation, etc. 

\section{Host Institution}

ICTLab of University of Science and Technology of Hanoi (USTH) is a joint international research laboratory in the field of Information and Communication Technology (ICT) between USTH and other French partners. The lab has researchers from USTH, Institute of Information Technology (IOIT) of Vietnam Academy of Science and Technology (VAST), Institut de Recherche pour le Dévelopment (IRD) and the University of La Rochelle, France. [1] \\
~\\
The ICTLab was created on December 1st, 2014, supported by USTH, French Embassy in Vietnam, 13 French high education institutes and universities (namely, USTH Consortium), and the Asian Development Bank (ADB). [1] \\
~\\
The researches of ICTLab focus on two main applicative axes: Cultural heritage preservation and promotion, and Environmental protection and Environmental risk management. Fundamental research programs targets common scientific activities, such as Modeling, Image Processing and Analysis, Machine Learning, Expert User Interaction, Geographical Information Systems, Information Retrieval, and Sensor . [1] \\
~\\
Regarding to these research activities, ICTLab is currently working on two main projects. The first project, SWARMS (Say and Watch: Automated image/sound Recognition for Mobile monitoring Systems), aims to achieve a flexible and real-time monitoring network, where device can used as both passive sensors and active transmitter that can send visual, voice, textual pieces of information to a monitoring system, so that stakeholder can analyze and forecast information. The other project, ARCHIVES (Analysis and Reconstruction of Catastrophes in History within Interactive Virtual Environments and Simulations), focuses on supporting historical research on past disasters with assistance of advance document image processing and analysis, information retrieval, machine learning, Geographical Information System (GIS) representation and agent-based computer modeling. \\

\section{Regional Climate Model system(RegCM)}

The Regional Climate Model system, also known as RegCM, 
\section{Limitation of RegCM}

\vspace*{1cm}
Placeholder

\section{Report organization}

This part must explain how the report is organized.
