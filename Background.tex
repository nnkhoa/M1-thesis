\chapter{Background}

\section{Parallel Computing}

As the name imply, parallel computing is about executing task simultaneously based on the given computer resource to find a solution to a complex computational problem, such as modeling and simulation. Unlike the traditional serial computation, where instructions are executed in succession one at a time, in parallelization, a problem are separated many different, disctint segments, each segment can be further decomposed into set of instructions. The distintion between these segments allows them to be performed on different processors in parallel fashion. The computation resource required for parallel computing, as stated previously, can be either a powerful computer with a multiple cores/threads CPU or a network of serveral computers, with or without dedicated GPUs. \\
~\\
The motivation behinds parallel computing is strongly related to real world problems, where many events, complementing each other, happen more or less simultaneously, however inside a chronological sequence. Because of how natural phenomena work and how complex they are, serial computing is not suited for heavy workload of modeling and simulation such events. Serialize computation is too time-consuming for this kind of tasks, as being only able to do one job at a time. Parallel computing, on the other hand, is clearly much more fitted for this line of work. By taking advantage of computers network, whether it is a local or remote one, tapping into parallel potential within modern computer hardwares(multi-core CPU, CUDA GPU) or even both of the above at the same time, parallelization provides the power needed to not only solving mathematical problems with high level of complexity that can be impossible to run on serial computing model, but also gaining better performance time, result in more computation tasks can be done, hence more efficient and profitable. \\
~\\
Current Limitation

\section{High Performance Computing}

\subsection{Cluster}

What is Cluster?

\subsection{Grid Computing}

What is Grid?

\section{Local Parallism}

